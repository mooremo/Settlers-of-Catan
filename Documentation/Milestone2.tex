\documentclass[12pt]{article}
\usepackage{geometry}                % See geometry.pdf to learn the layout options. There are lots.
\geometry{letterpaper}                   % ... or a4paper or a5paper or ... 

\usepackage[parfill]{parskip}    % Activate to begin paragraphs with an empty line rather than an indent
\usepackage{fullpage}
\usepackage{graphicx}
\usepackage{grffile}
\usepackage{listings}
\usepackage{hyperref}
\usepackage{tabularx} %tabular with stretch columns
\usepackage{enumerate}
\usepackage{pdfpages}
\usepackage{pdflscape}
\usepackage{color}
\usepackage{amsmath}
\usepackage{amsfonts}
\usepackage{amssymb}

% Index
\usepackage{makeidx}
\makeindex

\begin{document}



% Fancy custom title page from Jimmy Theis
\begin{titlepage}
\begin{center}
\includegraphics[width=0.15\textwidth]{Images/logo}\\[1cm]
\textsc{\LARGE Rose-Hulman Institute of Technology}\\[1.5cm]
\textsc{\Large Settlers of C\#tan}\\[1.0cm]
\textsc{\Large Milestone 2}\\[1.0cm]
\Large Liz Hines \hspace{0.2cm}*\hspace{0.2cm}  Matt Moore \hspace{0.2cm}*\hspace{0.2cm} Sam	Kim
\vfill
\large \today
\end{center}
\end{titlepage}

\newpage
{\small \tableofcontents}
\newpage

\section{Schedule}
Our schedule is viewable on the \href{https://github.com/mooremo/Settlers-of-Catan/issues/milestones}{Github issue tracker}. All features and tests are assigned to weekly sprints. Sprint 6, which coincides with week 10, is reserved for remaining burn down.

\section{Coding Standards}
We will follow the coding standards as outlined in the \href{http://goo.gl/XRoHg}{Microsoft C\# Coding Guidelines} and the \href{http://goo.gl/kuwHu}{Microsoft .NET Framework General Reference}.

\section{Software Configuration Management Plan}
\subsection{Reference Documents}
All documentation pertaining to the \emph{Settlers of C\#tan} can be found in the \href{https://github.com/mooremo/Settlers-of-Catan/tree/master/Documentation}{Documentation} folder of the teams \href{https://github.com/mooremo/Settlers-of-Catan}{Github repository}.

\section{Management}
\subsection{Responsibilities}
The software configuration will be the responsibility of the whole team. Each team member will perform a weekly code walk checking for and correcting mistakes in the configuration. The project advisor, Sriram Mohan, may suggest changes to the software configuration management.

\subsection{Organization}
\begin{enumerate}
\item The team will maintain a \href{https://github.com/mooremo/Settlers-of-Catan}{Github repository} that contains all of the project's code and documentation.
\item This repository will be available to the advisor.
\item Commit messages to the repository will be made in the imperative, present tense and be no longer than 140 characters.
\item The \href{https://github.com/mooremo/Settlers-of-Catan/tree/master/Documentation}{Documentation} folder will contain all project deliverables including milestones, status reports, and other documentation.
\item The \href{https://github.com/mooremo/Settlers-of-Catan/tree/master/Code/SettlersOfCatan}{Code} folder will contain all of the project's source code.
\item The \href{https://github.com/mooremo/Settlers-of-Catan/tree/master/Code/SettlersOfCatanTest}{Tests} folder will contain all of the project's test code.
\end{enumerate}

\section{Configuration Identification}
All project artifacts, source code, and test code will be placed under configuration control. 

Project artifacts are defined as the following:
\begin{itemize}
\item Milestones
\item Minutes
\item Agendas
\item Status Reports
\item Presentations
\end{itemize}

\section{Naming Conventions}
Title case should be used for all document names and no revision number should be contained in the name.

The naming conventions for source and test code are as described in the coding standards.

\section{Change Request Procedure}
\subsection{Requesting a Change}
Non-trivial changes are requested via the Github issue tracker. When creating a change request the required fields are:
\begin{itemize}
\item Title - A brief descriptive name.
\item Body - A detailed explanation of the requested change. If the change request is related to a bug, instructions to reproduce should also be included.
\end{itemize}

\subsection{Evaluating Requests}
After a team member reads a change request it is labeled as either bug, enhancement, or other and then discussed by the team. Requests are evaluated based on feasibility, time commitment, and scope.

\subsection{Approving or Disapproving Changes}
After the team has discussed a change request, it can be accepted by assigning to a milestone and a team member. If disapproved, it is closed and an explanatory comment is provided.

\subsection{Implementing Changes}
Changes are implemented in the assigned sprint by the assigned team member. That team member is responsible for closing the change request when completed.




\newpage
\section{Revision History}
Created 4/10/12 by Elizabeth Hines, Sam Kim, Matt Moore

\end{document}
